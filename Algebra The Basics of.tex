% Options for packages loaded elsewhere
\PassOptionsToPackage{unicode}{hyperref}
\PassOptionsToPackage{hyphens}{url}
%
\documentclass[
]{article}
\usepackage{amsmath,amssymb}
\usepackage{lmodern}
\usepackage{iftex}
\ifPDFTeX
  \usepackage[T1]{fontenc}
  \usepackage[utf8]{inputenc}
  \usepackage{textcomp} % provide euro and other symbols
\else % if luatex or xetex
  \usepackage{unicode-math}
  \defaultfontfeatures{Scale=MatchLowercase}
  \defaultfontfeatures[\rmfamily]{Ligatures=TeX,Scale=1}
\fi
% Use upquote if available, for straight quotes in verbatim environments
\IfFileExists{upquote.sty}{\usepackage{upquote}}{}
\IfFileExists{microtype.sty}{% use microtype if available
  \usepackage[]{microtype}
  \UseMicrotypeSet[protrusion]{basicmath} % disable protrusion for tt fonts
}{}
\makeatletter
\@ifundefined{KOMAClassName}{% if non-KOMA class
  \IfFileExists{parskip.sty}{%
    \usepackage{parskip}
  }{% else
    \setlength{\parindent}{0pt}
    \setlength{\parskip}{6pt plus 2pt minus 1pt}}
}{% if KOMA class
  \KOMAoptions{parskip=half}}
\makeatother
\usepackage{xcolor}
\IfFileExists{xurl.sty}{\usepackage{xurl}}{} % add URL line breaks if available
\IfFileExists{bookmark.sty}{\usepackage{bookmark}}{\usepackage{hyperref}}
\hypersetup{
  hidelinks,
  pdfcreator={LaTeX via pandoc}}
\urlstyle{same} % disable monospaced font for URLs
\setlength{\emergencystretch}{3em} % prevent overfull lines
\providecommand{\tightlist}{%
  \setlength{\itemsep}{0pt}\setlength{\parskip}{0pt}}
\setcounter{secnumdepth}{-\maxdimen} % remove section numbering
\ifLuaTeX
  \usepackage{selnolig}  % disable illegal ligatures
\fi

\author{}
\date{}

\begin{document}

\hypertarget{algebra-the-basics-of-it}{%
\section{Algebra, The Basics of it}\label{algebra-the-basics-of-it}}

In this text, we'll assume just basic Set Theory, involving the concepts
of relations, equivalence classes, Cartesian product and so on.

So... what is algebra about?

\hypertarget{main-concepts}{%
\subsection{Main Concepts}\label{main-concepts}}

\hypertarget{binary-operation}{%
\subsubsection{Binary operation}\label{binary-operation}}

\hypertarget{group}{%
\subsubsection{Group}\label{group}}

\hypertarget{ring}{%
\subsubsection{Ring}\label{ring}}

Think about the set \(\Z\) and the operations \(+\) and \(\times\) on
it. What properties do these operations satisfy?

We know that addition is associative and commutative and that it has a
neutral element denoted by \(0\) and an inverse element.

With respect to multiplication, it is associative and commutative, it
has a neutral element and the addition is distributive related to
multiplication.

A ring is a generalization of these properties that we already know. In
Algebra, we are going to study sets with two operations \(+\) and
\(\times\) that satisfy the properties elucidated above. These
operations are not necessarily the addition and multiplication that we
know for \(\Z\).

If, for a given ring, all these properties are satisfied except for the
commutative of multiplication, then we'll call it a non-commutative
ring.

In summary, a \textbf{commutative ring} \((A, +, \times)\) is a set
\(A\) with at least two elements, an operation denoted by \(+\) (called
addition) and an operation denoted by \(\times\) (called multiplication)
satisfying:

\begin{enumerate}
\def\labelenumi{\arabic{enumi}.}
\item
  Addition is associative, i.e.,
  \(\forall x,y,z \in A, (x+y)+z = x+(y+z)\).
\item
  Addition has a neutral element, i.e., \(\exists 0 \in A\) such that
  \(\forall x \in A, 0+x = x \land x+0 = x\).
\item
  Addition has an inverse element, i.e., \(\exists x \in A\) such that
  \(x+z = 0 \land z+x = 0\).
\item
  Addition is commutative, i.e., \(\forall x, y \in A, x+y = y+x\).
\item
  Multiplication is associative, i.e.,
  \(\forall x, y, z \in A, (x \times y) \times z = x \times (y \times z)\).
\item
  Multiplication has a neutral element, i.e., \(\exists 1 \in A\) such
  that \(\forall x \in A, 1 \times x = x \land x \times 1 = x\).
\item
  Multiplication is commutative, i.e.,
  \(\forall x, y \in A, x \times y = y \times x\).
\item
  Addition is distributive with respect to multiplication, i.e.,
  \(\forall x, y, z \in A, x \times (y + z) = x \times y + x \times z\).
\end{enumerate}

Where \(\land\) is a logical symbol for "and" and we may denote
\(a \times b\) by \(a \cdot b\) or simply \(ab\).

Some important remarks:

\begin{itemize}
\item
  The neutral element is unique, both for addition and multiplication.
\item
  The inverse element, with respect to addition, is unique.
\item
  The neutral element of addition has the following property:
  \(0 \cdot x = 0, \forall x \in A\).
\end{itemize}

\hypertarget{domain}{%
\subsubsection{\texorpdfstring{Domain }{Domain }}\label{domain}}

If the product of any two non-zero elements of a given ring \(K\) is
different than zero, then this ring is called a \textbf{domain} or an
\textbf{integral domain}.

\[\forall x, y \in K \setminus \{ 0 \}, \, x \times y \neq 0\]

E.g. \((\mathbb{Z}, +, \times)\) is a domain.

E.g. Let \(\mathbb{Z}[i] = \{ a+ bi \, |  \, a,b \in \mathbb{Z} \}\).
Then \((\mathbb{Z}[i], +, \times)\) is a domain called \textbf{Gaussian
integers}.

E.g. \textbf{Ring of integers modulo n.} Let \(n \in \mathbb{Z}_+\) and
define the relation \(\underset{n}{\equiv}\) as
\(a \underset{n}{\equiv} b \iff a - b\) is a multiple of \(n\). Notice
that \(\underset{n}{\equiv}\) is an equivalence class. The equivalence
class is denoted by \(\bar{a} = \{ a+kn, k \in \mathbb{Z} \}\). We write
\(\mathbb{Z}/n \mathbb{Z}\) as the set of equivalence classes modulo
\(n\), where
\(\mathbb{Z}/n \mathbb{Z} = \{ \bar{0}, \bar{1}, \ldots, \overline{n-1} \}\).

If \(I \subseteq A\), \(I\) not empty, then \(I\) is called an
\textbf{ideal} of \(A\) if

\begin{itemize}
\item
  \(x+y \in I, \forall x, y \in I\).
\item
  \(ax \in I, \forall x \in I, \forall a \in A\).
\end{itemize}

E.g. \(n \mathbb{Z} := \{zn \, | \, z \in \mathbb{Z} \}\) is an ideal
(where \(n\) is a non-negative integer).

This concept allows us to make constructions analogous to the ring of
integers modulo \(n\). In general, an ideal is not a ring because it
usually lacks a neutral element.

E.g. \textbf{Quotient ring modulo an ideal}. Define
\(a \equiv b \mod I \iff a- b \in I\). Then
\(\bar{a} := \{ b \in A \, | \, b \equiv a \mod I \} = \{ a + c \, | \, c \in I \} = a + I\).

\hypertarget{field}{%
\subsubsection{Field}\label{field}}

If every non-zero element of a given ring has a multiplicative inverse,
then this ring is called a \textbf{field}. Symbolically,

\[\forall x \in K \setminus \{0 \}, \, \exists y \in K : x \times y = 1\]

This inverse is also unique.

Please take notice that this property is stronger than the property that
defines a domain. That means that is \(K\) is a field, then \(K\) is
also a domain. Although not all domains are fields, if the domain has a
finite number of elements, then it is a field.

E.g.
\((\mathbb{Q},+,\times), (\mathbb{R},+,\times), (\mathbb{C},+,\times)\)
are fields.

\hypertarget{direct-product}{%
\subsubsection{Direct product}\label{direct-product}}

Given two rings \((A_1, +_1, \cdot_1)\) and \((A_2, +_2, \cdot_2)\),
it's possible to construct a new ring defining:

\begin{itemize}
\item
  The set
  \(A_1 \times A_2 := \{ (a_1, a_2) \, | \, a_1 \in A_1, a_2 \in A_2 \}\)
\item
  Addition:
  \((a_1, a_2) + (a_1', a_2') := (a_1 +_1 a_1', a_2 +_2 a_2')\)
\item
  Multiplication:
  \((a_1, a_2) \times (a_1', a_2') := (a_1 \cdot_1 a_1', a_2 \cdot_2 a_2')\).
\end{itemize}

Then \((A_1 \times A_2, +, \cdot)\) is a ring called \textbf{direct
product} of \(A_1\) and \(A_2\).

\hypertarget{lattice}{%
\subsubsection{Lattice}\label{lattice}}

\hypertarget{module}{%
\subsubsection{Module}\label{module}}

\hypertarget{algebra}{%
\subsubsection{Algebra}\label{algebra}}

\hypertarget{polynomial-rings}{%
\subsection{Polynomial Rings}\label{polynomial-rings}}

\end{document}
