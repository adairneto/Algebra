\documentclass[12pt,a4paper]{article}
\usepackage[american]{babel}
\usepackage{amsmath}
\usepackage{amsfonts}
\usepackage{amssymb}
\usepackage[utf8x]{inputenc}
\setlength{\parindent}{1.5em}
\setlength{\parskip}{0.5em}
\usepackage{indentfirst}
\usepackage{float}
\usepackage[sfdefault]{FiraSans} %% option 'sfdefault' activates Fira Sans as the default text font
\usepackage[T1]{fontenc}
\renewcommand*\oldstylenums[1]{{\firaoldstyle #1}}
\usepackage[titles]{tocloft}
\renewcommand{\cftdot}{}
\usepackage[colorlinks=true, allcolors=magenta]{hyperref}
\usepackage{url}
\usepackage[dvipsnames]{xcolor}
\usepackage{tcolorbox}
\tcbuselibrary{theorems}
\newtcbtheorem[number within=section]{thm}{Theorem}%
{colback=white!5,colframe=SpringGreen!35!black,fonttitle=\bfseries}{theorem}
\newtcbtheorem[number within=section]{exemple}{Example}%
{colback=white!5,colframe=Magenta!35!black,fonttitle=\bfseries}{theorem}
\newtcbtheorem[number within=section]{defn}{Definition}%
{colback=white!5,colframe=SpringGreen!35!black,fonttitle=\bfseries, arc=0mm}{theorem}
\newenvironment{proof}{\paragraph{Proof:}}{\hfill$\blacksquare$}
%\usepackage{sectsty}
%\subsectionfont{\color{RubineRed}}
\usepackage{hyperref}
\hypersetup{
    colorlinks=true,
    linkcolor=magenta,
    filecolor=cyan,      
    urlcolor=magenta,
    pdftitle={Algebra: The Basics of It},
    pdfpagemode=FullScreen,
    }
\author{\href{https://adairneto.github.io/}{Adair Antonio da Silva Neto}}
\title{Algebra: The Basics of It}

\begin{document}

\clearpage\maketitle
\thispagestyle{empty}

\newpage

\tableofcontents

\newpage
\clearpage
\setcounter{page}{1}

\section{Introduction}

In this text, we'll assume just basic Set Theory, involving the concepts of relations, equivalence classes, Cartesian product and so on.

So... what is algebra about?

% \section{Binary operation}\label{binary-operation}

% \section{Group}\label{group}

\section{Definitions}

\subsection{Ring}\label{ring}

Think about the set of integers \(\mathbb{Z} \) and the operations \(+\) and \(\times\) on it. What properties do these operations satisfy?

We know that addition is associative and commutative and that it has a neutral element denoted by \(0\) and an inverse element.

With respect to multiplication, it is associative and commutative, it has a neutral element and the addition is distributive with respect to multiplication.

A ring is a generalization of these properties that we already know. In this section, we are going to study sets with two operations \(+\) and \(\times\) that satisfy the properties elucidated above. These operations are not necessarily the addition and multiplication that we know for \(\mathbb{Z}\).

\begin{defn}{Ring}
AA \textbf{commutative ring} \((A, +, \times)\) is a set \(A\) with at least two elements, an operation denoted by \(+\) (called addition) and an operation denoted by \(\times\) (called multiplication) satisfying:
\begin{enumerate}
\item[A1)] Addition is \textbf{associative}: \(\forall x,y,z \in A, (x+y)+z = x+(y+z)\)
\item[A2)] Addition has a \textbf{neutral element}: \(\exists 0 \in A\) such that $\forall x \in A, 0+x = x$ and $x+0 = x$.
\item[A3)] Addition has an \textbf{inverse element}: \(\forall x \in A, \exists z \in A\) such that $x+z = 0$ and $z+x = 0$.
\item[A4)] Addition is \textbf{commutative}: \(\forall x, y \in A, x+y = y+x\).
\item[M1)] Multiplication is \textbf{associative}:  \(\forall x, y, z \in A, (x \times y) \times z = x \times (y \times z)\).
\item[M2)] Multiplication has a \textbf{neutral element}: \(\exists 1 \in A\) such that $\forall x \in A, 1 \times x = x$ and $x \times 1 = x$. $1$ is also called \textbf{identity}.
\item[M3)] Multiplication is \textbf{commutative}: \(\forall x, y \in A, x \times y = y \times x\).
\item[D)] Addition is \textbf{distributive} with respect to multiplication:  \(\forall x, y, z \in A, x \times (y + z) = x \times y + x \times z\).
\end{enumerate}

We may denote \(a \times b\) by \(a \cdot b\) or simply \(ab\) and context will elucidate.
\end{defn}

If, for a given ring, all these properties are satisfied except for the commutative of multiplication, then we'll call it a non-commutative ring.

Some important remarks:
\begin{itemize}
\item The neutral element is unique, both for addition and multiplication.
\item The inverse element, with respect to addition, is unique.
\item The neutral element of addition has the following property: \(0 \cdot x = 0, \forall x \in A\).
\end{itemize}

We also define a \textbf{subring} of a ring as the subset which is closed under the operations of addition, subtraction, and multiplication and also contains the element $1$.

Before heading to some examples, let give two important definitions.

\subsection{Domain}\label{domain}

If the product of any two non-zero elements of a given ring \(D\) is different than zero, then this ring is called a \textbf{domain} or an \textbf{integral domain}.
\[\text{M4) } \forall x, y \in D \setminus \{ 0 \}, \, x \times y \neq 0\]

Hence, if $D$ is an integral domain, $a,b,c \in D$ and $ab = ac$, then $a \neq 0$ implies $b=c$. This is called \textbf{cancellation law}.

An immediate example of a domain is $(\mathbb{Z}, +, \times)$.

\subsection{Field}\label{field}

If every non-zero element of a given ring has a multiplicative inverse, then this ring is called a \textbf{field}. Symbolically,
\[\text{M4') }\forall x \in K \setminus \{0 \}, \, \exists y \in K : x \times y = 1\]
This inverse is also unique and the inverse of $x$ is denoted by $x^{-1}$.

Please take notice that this property is stronger than the property that defines a domain. That means that is \(K\) is a field, then \(K\) is also a domain. Although not all domains are fields, if the domain has a finite number of elements, then it is a field.

E.g. \((\mathbb{Q},+,\times), (\mathbb{R},+,\times), (\mathbb{C},+,\times)\) are fields.

\section{Examples}

\subsection{Gaussian Integers}

Let \(\mathbb{Z}[i] = \{ a+ bi \, |  \, a,b \in \mathbb{Z} \}\). Then \((\mathbb{Z}[i], +, \times)\) is a domain called \textbf{Gaussian integers}. This idea can be generalized by defining a subring generated by $n$ as $\mathbb{Z}[n]$.

\subsection{Direct product}\label{direct-product}

Given two rings \((A_1, +_1, \cdot_1)\) and \((A_2, +_2, \cdot_2)\), it's possible to construct a new ring defining:

\begin{itemize}
\item
  The set
  \(A_1 \times A_2 := \{ (a_1, a_2) \, | \, a_1 \in A_1, a_2 \in A_2 \}\)
\item
  Addition:
  \((a_1, a_2) + (a_1', a_2') := (a_1 +_1 a_1', a_2 +_2 a_2')\)
\item
  Multiplication:
  \((a_1, a_2) \times (a_1', a_2') := (a_1 \cdot_1 a_1', a_2 \cdot_2 a_2')\).
\end{itemize}

Then \((A_1 \times A_2, +, \cdot)\) is a ring called \textbf{direct product} of \(A_1\) and \(A_2\).

\subsection{Ring of integers modulo n}

Let \(n \in \mathbb{Z}_+\) and define the relation \(\underset{n}{\equiv}\) as: given $a,b \in \mathbb{Z}$,
\[ a \underset{n}{\equiv} b \iff a - b \text{ is a multiple of } n \]
And we say that $a$ is congruent to $b$ modulo $n$. Notice that \(\underset{n}{\equiv}\) is an equivalence class.

The equivalence class of $a$ is denoted by \(\bar{a} = \{ b \in \mathbb{Z} \, | \, b \underset{n}{\equiv} a \} = \{ a+kn, k \in \mathbb{Z} \}\).

We also write \(\mathbb{Z}/n \mathbb{Z}\) to denote the set of equivalence classes modulo \(n\), i.e. \(\mathbb{Z}/n \mathbb{Z} = \{ \bar{0}, \bar{1}, \ldots, \overline{n-1} \}\). And we define the following compositions over it:
\begin{equation*}
\begin{aligned}
\underset{n}{\oplus}:  \mathbb{Z}/n \mathbb{Z} & \times \mathbb{Z}/n \mathbb{Z} &\to \mathbb{Z}/n \mathbb{Z}\\
&(\bar{x}, \bar{y}) &\mapsto \overline{x+y}
\end{aligned}
\end{equation*}
\begin{equation*}
\begin{aligned}
\underset{n}{\odot}: \mathbb{Z}/n \mathbb{Z} & \times \mathbb{Z}/n \mathbb{Z} &\to \mathbb{Z}/n \mathbb{Z}\\
&(\bar{x}, \bar{y}) &\mapsto \overline{xy}
\end{aligned}
\end{equation*}

Then $(\mathbb{Z}/n \mathbb{Z}, \underset{n}{\oplus}, \underset{n}{\odot})$ is a ring, where the neutral element for $\underset{n}{\oplus}$ is the class $\bar{0}$, the neutral element for $\underset{n}{\odot}$ is $\bar{1}$, and the inverse of $\bar{x}$ with respect to $\underset{n}{\oplus}$ is the class $\overline{-x}$.

\subsection{Ideal}

Before going to next example, a simple definition is necessary.

\begin{defn}{Ideal}
IIf \(I \subseteq A\), $I \neq \emptyset$, then \(I\) is called an \textbf{ideal} of \(A\) if
\begin{itemize}
\item
  \(x+y \in I, \forall x, y \in I\).
\item
  \(ax \in I, \forall x \in I, \forall a \in A\).
\end{itemize}
\end{defn}

For example, \(n \mathbb{Z} := \{zn \, | \, z \in \mathbb{Z} \}\) is an ideal of the ring of integers (where \(n\) is a non-negative integer).

This concept allows us to make constructions analogous to the ring of integers modulo \(n\). In general, an ideal is not a ring because it usually lacks a neutral element.

\subsection{Quotient ring modulo an ideal}

Given a ring $A$ and $I$ an ideal of $A$, define the following congruence relation: for $a, b \in A$,
\[a \equiv b \mod I \iff a- b \in I  \]
Which is also an equivalence relation.

If $a \in A$, then its class of equivalence modulo $I$ is given by \(\bar{a} := \{ b \in A \, | \, b \equiv a \mod I \} = \{ a + c \, | \, c \in I \}\). We denote $A/I$ the set of equivalence classes modulo $I$, and we define the following operations over it: for $\bar{x}, \bar{y} \in A/I$, 
\[
\bar{x} \underset{I}{\oplus} \bar{y} := \overline{x+y} \text{ and } \bar{x} \underset{I}{\odot} \bar{y} := \overline{x\cdot y}
\]
These operations are well defined and $(A/I, \underset{I}{\oplus},  \underset{I}{\odot})$ is a ring called \textbf{quotient ring modulo an ideal}.

% \section{Polynomial Rings}\label{polynomial-rings}

% \section{Lattice}\label{lattice}

% \section{Module}\label{module}

% \section{Algebra}\label{algebra}

\nocite{*}
\bibliographystyle{alpha}
\bibliography{algebra.bib}

\end{document}